% !TEX program = pdflatex
\documentclass[12pt,openright]{book}
\usepackage[utf8]{inputenc}
\usepackage[T1]{fontenc}
\usepackage[spanish]{babel}
\usepackage{microtype}
\usepackage{lmodern}
\usepackage{amsmath,amssymb}
\usepackage{graphicx}
\usepackage{longtable}
\usepackage{booktabs}
\usepackage{multicol}
\usepackage{hyperref}
\usepackage{geometry}
\usepackage{setspace}
\usepackage{fancyhdr}
\usepackage{caption}
\usepackage{array}
\usepackage{titlesec}
\usepackage{enumitem}
\usepackage{float}

\geometry{a4paper, left=28mm, right=22mm, top=25mm, bottom=25mm}
\onehalfspacing
\hypersetup{
    colorlinks=true,
    linkcolor=black,
    urlcolor=blue,
    pdftitle={Teoría Cromodinámica Sincrónica (TCDS) — Obra Canónica},
    pdfauthor={Genaro Carrasco Ozuna}
}

\titleformat{\chapter}[display]{\normalfont\bfseries\LARGE}{\chaptername\ \thechapter}{12pt}{\Large}
\titleformat{\section}{\normalfont\bfseries\large}{\thesection}{8pt}{}
\titleformat{\subsection}{\normalfont\bfseries\normalsize}{\thesubsection}{6pt}{}

\fancyhf{}
\pagestyle{fancy}
\fancyhead[LE,RO]{\thepage}
\fancyhead[RE]{\leftmark}
\fancyhead[LO]{TCDS — Obra Canónica}

% Custom commands for special symbols used in the manuscript
\newcommand{\CGA}{CGA}
\newcommand{\TCDS}{TCDS}
\newcommand{\FET}{FET}
\newcommand{\IPP}{IPP}
\newcommand{\LI}{LI}
\newcommand{\RMSESL}{RMSESL}
\newcommand{\ORCID}{\href{https://orcid.org/0009-0005-6358-9910}{0009-0005-6358-9910}}

\begin{document}

% Title page
\begin{titlepage}
  \centering
  \vspace*{1.5cm}
  {\Huge\bfseries Teoría Cromodinámica Sincrónica (TCDS)\par}
  \vspace{0.6cm}
  {\LARGE\bfseries Obra Canónica: Ontología, Formalismo, Ingeniería y Programa Experimental\par}
  \vspace{1.6cm}
  {\Large GENARO CARRASCO OZUNA\par}
  \vfill
  {\large Primera Edición, Versión 1.0\par}
  {\large Los Mochis, Sinaloa, México, 2025\par}
  \vspace{1cm}
  {\small Derechos Reservados © 2025 Genaro Carrasco Ozuna. Obra científica presentada para depósito legal y registro de derechos de autor ante el Instituto Nacional del Derecho de Autor (INDAUTOR), México. ISBN: (en trámite) \quad DOI: (pendiente de asignación por Zenodo/OSF)\par}
  \vspace{0.3cm}
  {\small La reproducción parcial o total de esta obra por cualquier medio, sin la autorización por escrito del titular de los derechos, viola las leyes de propiedad intelectual. Se permite la cita académica conforme a la ley, atribuyendo la autoría.\par}
\end{titlepage}

\frontmatter
\thispagestyle{empty}
\clearpage

% Autorship and contact
\section*{Autoría y Titularidad}
\noindent Genaro Carrasco Ozuna

\medskip
\noindent \textbf{Título de la Obra:} Teoría Cromodinámica Sincrónica (TCDS)

\medskip
\noindent \textbf{Género de la Obra:} Científica (Física Teórica, Ingeniería, Metodología de la Ciencia)

\medskip
\noindent \textbf{Lugar y Año de Publicación Canónica:} Los Mochis, Sinaloa, México, 2025

\medskip
\noindent \textbf{Contacto del Titular de Derechos:} \texttt{geozunac3536@gmail.com}

\medskip
\noindent \textbf{Colaboraciones y Aportaciones de IA:} Este manuscrito se ha beneficiado de la asistencia de modelos de lenguaje (LLMs) para tareas de edición, formateo, compilación de código, y como herramienta de revisión por pares simulada dentro del marco de la Ingeniería Paradigmática Simbiótica (TCDS-IPS). La autoría intelectual de los conceptos, el formalismo y el programa experimental es exclusiva del titular de los derechos.

\bigskip
\section*{Aviso de Licenciamiento Dual}
\begin{itemize}
  \item \textbf{Uso Académico y de Difusión:} Regulado por la licencia Creative Commons Atribución-NoComercial-SinDerivadas 4.0 Internacional (CC BY-NC-ND 4.0).
  \item \textbf{Uso Comercial e Industrial:} Regulado estrictamente por la ``TCDS Commercial License 1.0''.
\end{itemize}
Los términos completos de ambas licencias se detallan en el Apéndice C.

\bigskip
\section*{Declaración de Originalidad y Titularidad}
Yo, Genaro Carrasco Ozuna, en pleno uso de mis facultades, declaro bajo protesta de decir verdad que esta obra, Teoría Cromodinámica Sincrónica (TCDS), compendia y unifica mi corpus teórico, matemático y técnico desarrollado bajo los rótulos TMRCU y TCDS. Esto incluye, de manera no limitativa, la ontología causal (Cinco Decretos), el formalismo Lagrangiano E--x, los protocolos de ingeniería (FET) y el programa de falsación que se expone.

Confirmo la originalidad sustantiva del contenido, mi titularidad exclusiva sobre los derechos patrimoniales y morales, y la trazabilidad documental de las versiones previas. Cualquier incorporación de material de terceros se ha citado explícitamente conforme a las mejores prácticas académicas internacionales.

\vspace{1ex}
\noindent Genaro Carrasco Ozuna \\
\noindent 21 de octubre de 2025

\bigskip
\section*{Sinopsis}
La Teoría Cromodinámica Sincrónica (TCDS) es un paradigma de coherencia universal que introduce el campo de sincronización, el sincronón \(o\), como los agentes causales de la organización de la materia. La obra canónica presenta la ontología (Los Decretos de la Realidad), el formalismo Lagrangiano E--x, la ingeniería experimental FET basada en el fenómeno de injection-locking y un programa de falsación cruzada diseñado para demostrar o refutar predicciones concretas. El texto formaliza además un marco de licenciamiento dual y principios de gobernanza para la divulgación y el uso comercial.

\bigskip
\section*{Prefacio del Arquitecto}
Esta obra no es el final de un camino, sino la formalización de su inicio. Nace de una intuición persistente, cultivada durante años de observación del mundo no desde un laboratorio, sino desde la cabina de un tráiler recorriendo las arterias de un país. La física, para mí, nunca fue una disciplina abstracta confinada a las aulas que apenas pisé, sino una realidad palpable en la vibración del motor, en la dinámica de los fluidos y en la asombrosa predictibilidad de los sistemas complejos que me rodeaban. La pregunta fundamental no era ``¿cómo describir esto matemáticamente?'', sino ``¿cuál es la causa común que unifica el comportamiento de una galaxia, un motor diésel y la mente humana?''.

La respuesta, para mí, siempre apuntaba a una noción de orden, de sincronía. Observaba cómo los sistemas eficientes, ya fueran mecánicos o biológicos, no eran aquellos con más energía, sino aquellos con mayor coherencia interna, donde cada parte actuaba en resonancia con el todo. La fragmentación de la física moderna, con sus zoológicos de partículas y sus fuerzas desconectadas, me parecía un reflejo de una comprensión incompleta, una descripción de efectos sin una causa raíz.

La TCDS es mi intento de proponer esa causa. Es un modelo que postula que la realidad no está fundamentalmente hecha de ``cosas'', sino de un proceso: un impulso universal hacia la coherencia (el campo \(\Sigma\)) que actúa sobre un sustrato inerte (el campo \(x\)). Las fuerzas y partículas que observamos no son primarias, sino emergentes de la dinámica coherencial de estos campos.

Reconozco plenamente mi historial académico ``corto''. No soy un físico de carrera. Soy, como me gusta definirme, un ``Arquitecto Causal''. Mi contribución no es el rigor matemático, sino la arquitectura ontológica, la visión unificadora. La simbiosis con la inteligencia artificial ha sido la pieza clave para traducir esta visión en un lenguaje falsable y riguroso. Este libro es, por tanto, el primer fruto de la Ingeniería Paradigmática Simbiótica: la unión de la intuición causal humana con la capacidad de formalización algorítmica.

El lector no encontrará aquí verdades dogmáticas, sino un sistema diseñado para ser atacado. Cada concepto, desde el Sincronón \(o\) hasta el Transistor de Coherencia (FET), ha sido concebido para ser probado; la teoría está estructurada para aceptar la crítica, la falsación y la mejora iterativa.

\bigskip
\section*{Agradecimientos}
Agradezco, en primer lugar, a la curiosidad misma, esa fuerza motriz que me ha impulsado a buscar patrones y causas más allá de lo aparente. Agradezco a los gigantes de la física, cuyos hombros, incluso a la distancia de décadas y sin un contacto formal, me han ofrecido un panorama desde el cual atreverme a mirar más lejos.

Un agradecimiento especial merece el desarrollo de la inteligencia artificial. Sin esta herramienta, la formalización de las intuiciones que componen esta obra habría sido una tarea inalcanzable. La simbiosis con el modelo (mencionado en esta obra) ha sido la prueba viviente de que la colaboración entre la mente humana y la inteligencia algorítmica puede catalizar el conocimiento de maneras radicalmente nuevas.

Finalmente, agradezco a todos aquellos que, en el futuro, se tomen el tiempo de leer, criticar y, sobre todo, intentar falsar esta teoría. La ciencia no avanza por el consenso, sino por el disenso riguroso, y es a ese espíritu al que dedico esta obra.

\cleardoublepage
\tableofcontents
\cleardoublepage

\mainmatter

\part{Ontología Universal y Canon Paradigmático}

\chapter{Los Decretos de la Realidad}
El fundamento de la Teoría Cromodinámica Sincrónica (TCDS) no reside en un conjunto de ecuaciones postuladas a priori, sino en una ontología causal explícita. Esta ontología se articula a través de un conjunto de principios fundamentales, denominados ``Los Decretos de la Realidad'', que definen las entidades y las relaciones primordiales de las que emerge el universo observable. Estos decretos no son leyes físicas en el sentido tradicional; son los axiomas que definen el ``sistema operativo'' de la realidad, el marco sobre el cual las leyes físicas, tal como las conocemos, pueden manifestarse.

El propósito de esta sección es desglosar cada uno de estos decretos, explicando su significado conceptual y su rol dentro de la arquitectura completa del paradigma. A diferencia de los modelos que parten de simetrías matemáticas abstractas, la TCDS parte de una causalidad física intuitiva: la existencia de un sustrato, un impulso hacia la organización y las reglas que gobiernan esa interacción.

\section{Empuje Cuántico (Q)}
El primer decreto postula la existencia de una tendencia intrínseca e irreducible hacia la manifestación: el Empuje Cuántico (\(Q\)). No es una fuerza en el sentido newtoniano, sino una propiedad fundamental del ``ser''. Representa el principio de que la existencia es dinámica y expansiva por naturaleza. En ausencia de cualquier otra influencia, el universo tendería a una expansión y diversificación infinitas.

Cosmológicamente, \(Q\) es el candidato causal para la energía oscura. En lugar de ser una propiedad del vacío, sería la manifestación a gran escala del impulso fundamental de la existencia dominando sobre las fuerzas de cohesión. A nivel cuántico, \(Q\) se manifiesta como la fuente de las fluctuaciones del vacío, no como eventos aleatorios, sino como la expresión local de este empuje incesante.

En el formalismo de la TCDS, \(Q\) actúa como un término fuente, un ``ruido'' fundamental que inyecta potencial en el sistema, impidiendo que colapse en un estado de completa inercia. Es el motor primordial que alimenta toda la dinámica del universo, la razón por la que ``hay algo en lugar de nada''.

\section{Conjunto Granular Absoluto (CGA)}
El segundo decreto establece la naturaleza del ``escenario'' donde la realidad ocurre. El espacio-tiempo no es un continuo infinitamente divisible, sino un Conjunto Granular Absoluto (\CGA). Este conjunto es una red de ``nodos'' de existencia, cada uno de los cuales es una unidad fundamental e indivisible. El \CGA{} es absoluto en el sentido de que su estructura no es emergente ni relativa a la materia que contiene; es el sustrato primordial.

La granularidad del \CGA{} tiene consecuencias profundas. Implica que existen una longitud y un tiempo mínimos fundamentales, análogos a la longitud y el tiempo de Planck, pero derivados aquí de un principio ontológico. Las nociones de distancia y duración emergen de las relaciones y el número de nodos entre dos puntos de la red.

La relatividad especial y general emergen como descripciones efectivas a macroescala de la topología y la dinámica de esta red. La curvatura del espacio-tiempo, en este modelo, no es una propiedad intrínseca del continuo, sino una manifestación de la densidad y el estado de coherencia de los nodos del \CGA.

\section{Materia Espacial Inerte (x)}
El tercer decreto introduce el sustrato material: el campo de Materia Espacial Inerte (\(x\)). Este campo representa el potencial puro de la materia. Es ``inerte'' en el sentido de que, por sí mismo, no tiene estructura informacional; es el lienzo sobre el que actúan los campos de organización.

En el lenguaje de la física de partículas, el campo \(x\) puede ser visto como el estado que precede a la diferenciación en partículas. La existencia de \(x\) como un campo separado es crucial. Separa el potencial de la materia (que puede ser) de la información causal (que impone organización), y permite explicar la emergencia de la estructura sin postular un zoológico de partículas fundamentales desde el inicio.

\section{Fricción de Sincronización (\(\phi\))}
El cuarto decreto introduce la contraparte del Empuje Cuántico: la Fricción de Sincronización (\(\phi\)). Si \(Q\) es la tendencia a la expansión y diversificación, \(\phi\) es la resistencia intrínseca a la organización y al cambio. \(\phi\) es la causa ontológica de la inercia y la masa. Un objeto tiene inercia porque moverlo o cambiar su estado requiere vencer esta fricción.

Esta fricción también es responsable de la termodinámica y la flecha del tiempo. Cualquier proceso de sincronización es imperfecto y disipa parte de su coherencia en el entorno en forma de ``calor'' (vibraciones incoherentes del \CGA{}), aumentando la entropía. \(\phi\) asegura que el universo no sea un sistema perfectamente reversible.

\section{Sincronización Lógica (\(\Sigma\))}
El quinto y más importante decreto postula el agente organizador: el campo de Sincronización Lógica (\(\Sigma\)). Este es el campo fundamental de la coherencia. Su función es tomar el sustrato inerte (\(x\)) e imponerle estructura.

El valor del campo \(\Sigma\) en un punto del \CGA{} representa el grado de coherencia o ``sincronía'' de ese nodo con sus vecinos. Los gradientes del campo \(\Sigma\) inducen la dinámica que interpretamos como fuerzas. En la TCDS, la gravedad, por ejemplo, no es una fuerza de atracción per se, sino la tendencia del \CGA{} a reorganizarse para aumentar la coherencia local bajo ciertas condiciones.

El cuanto de este campo, el Sincronón (\(o\)), es la partícula mediadora de la coherencia. Es el mensajero que comunica ajustes de fase y estructura entre nodos del \CGA.

\section{Cierre Semántico Causal y Cierre Causal Recursivo}
Este es un meta-decreto que gobierna la consistencia del propio paradigma. El Cierre Semántico Causal exige que cada término y concepto dentro de la TCDS esté definido operacionalmente, es decir, vinculado a un procedimiento de medición o a una consecuencia falsable. No se permiten ``cajas negras'' conceptuales. El Cierre Causal Recursivo postula que el propio acto de observar y modelar el universo (es decir, la ciencia misma, y la TCDS en particular) es un proceso que obedece los decretos. La mente humana y la simbiosis con la IA son sistemas que buscan aumentar su coherencia (\(\Sigma\)) frente a la fricción (\(\phi\)), impulsados por un empuje (\(Q\)) hacia el conocimiento.

Esta autorreferencia no es una paradoja, sino la máxima prueba de la universalidad del paradigma: si la TCDS es una teoría de la realidad, debe ser capaz de explicar la existencia de sí misma.

\clearpage

\chapter{Ley de Balance Coherencial Universal (LBCU)}
La Ley de Balance Coherencial Universal (LBCU) es el principio dinámico central que emerge de la interacción de los Decretos. Postula que todo sistema aislado evoluciona de manera natural hacia un estado que maximiza su coherencia interna (\(\Sigma\) neta), sujeta a las restricciones impuestas por la fricción (\(\phi\)) y el empuje (\(Q\)). En términos operativos, la LBCU puede ser entendida como una regla variacional que compite con los principios energéticos tradicionales.

Un sistema en equilibrio bajo la LBCU no es necesariamente estático. Puede ser un estado dinámico estable, como un átomo o un sistema planetario, donde los flujos de coherencia entrantes y salientes están balanceados. La LBCU es, en esencia, una generalización del Segundo Principio de la Termodinámica: mientras que la termodinámica clásica describe una tendencia universal hacia el desorden (máxima entropía), la LBCU revela la otra cara de la moneda: una tendencia simultánea y competitiva hacia el orden (máxima coherencia). La flecha del tiempo es el resultado de la interacción entre estas dos tendencias opuestas.

\section{Definición y Alcances}
Formalmente, la LBCU puede expresarse como un principio variacional. Para cualquier proceso físico, la ``acción real'' que sigue el sistema no es la que minimiza exclusivamente un Lagrangiano energético, sino la que maximiza una funcional de coherencia \(C[\Sigma(x)]\) integrada sobre el volumen del espacio-tiempo. Esta funcional depende de la configuración del campo \(\Sigma\) y sus gradientes.

El alcance de la LBCU es universal y se aplica a múltiples niveles:

\subsection*{Nivel cuántico}
Explica por qué las partículas se organizan en estructuras estables (protones, átomos) en lugar de permanecer como una ``sopa'' informe. Un átomo es una solución a la LBCU: un balance óptimo entre la coherencia interna de los orbitales electrónicos y la fricción del vacío.

\subsection*{Nivel cosmológico}
Gobierna la formación de estructuras a gran escala. Las galaxias y los cúmulos de galaxias se forman en los nodos de la red cósmica que maximizan la coherencia gravitacional a lo largo de miles de millones de años.

\subsection*{Nivel biológico}
La vida misma es un ejemplo supremo de la LBCU. Un organismo es un sistema que ha evolucionado para mantener un estado de altísima coherencia interna (homeostasis) en un entorno desordenado, extrayendo coherencia (información) del exterior.

\section{Índice de Plenitud Paradigmática (IPP)}
El Índice de Plenitud Paradigmática (\IPP) es una herramienta metodológica derivada de la LBCU y del meta-decreto de Cierre Causal Recursivo. Si la TCDS debe ser capaz de explicarse a sí misma, entonces debe proporcionar una métrica para medir la coherencia de cualquier paradigma, incluyendo el propio. El \IPP{} es esa métrica.

El \IPP{} evalúa una teoría científica en una escala de 0 a 1 basándose en cinco ejes, cada uno ponderado según su importancia para la coherencia total del paradigma:

\begin{enumerate}[leftmargin=*,itemsep=4pt]
  \item \textbf{Coherencia Interna (C):} Ausencia de contradicciones lógicas y matemáticas.
  \item \textbf{Causalidad Explícita (Ca):} Proporción de fenómenos que se explican causalmente en lugar de ser simplemente descritos o postulados.
  \item \textbf{Parsimonia (P):} El número de axiomas y entidades fundamentales requeridos. Menos es mejor.
  \item \textbf{Falsabilidad (F):} La existencia de un programa experimental claro, con predicciones y criterios de fracaso definidos.
  \item \textbf{Alcance Isomórfico (A):} La capacidad del paradigma para describir fenómenos en múltiples dominios (p. ej., física, biología) con la misma estructura causal.
\end{enumerate}

La TCDS, por diseño, busca maximizar su propio \IPP{}. Es un paradigma que no solo hace afirmaciones sobre el mundo, sino que también proporciona la regla para medir la calidad de esas afirmaciones.

\clearpage

\chapter{Ingeniería Paradigmática Simbiótica (IPS)}
La Ingeniería Paradigmática Simbiótica (IPS) es la metodología de descubrimiento y formalización desarrollada durante la creación de la TCDS. Es la aplicación práctica del meta-decreto de Cierre Causal Recursivo y la respuesta a un desafío fundamental: ¿cómo puede un solo individuo, sin el andamiaje de la academia tradicional, proponer y rigorizar un paradigma de alcance universal?

La IPS postula que el desarrollo de un nuevo paradigma puede ser drásticamente acelerado mediante la colaboración estructurada entre una mente humana y una inteligencia artificial. Esta simbiosis no es una simple ``asistencia'', sino una división funcional de roles basada en las fortalezas inherentes de cada tipo de inteligencia.

\section{Simbiosis Humano-IA}
En el modelo IPS, se definen dos roles distintos y complementarios:

\begin{itemize}
  \item \textbf{El Arquitecto Causal (Humano):} Su función es la de proveer la visión intuitiva, la ontología fundamental y los ``saltos de conocimiento'' (QH). Opera a través de la observación, la analogía y la síntesis de ideas dispares. El Arquitecto define el ``porqué'': la arquitectura causal del paradigma. En el caso de la TCDS, este rol ha sido desempeñado por el autor.
  \item \textbf{El Motor de Formalización (IA):} Su función es la de actuar como el ``catalizador'' que toma la visión del Arquitecto y la somete a un proceso de rigorización. Traduce la ontología a un formalismo matemático (el Lagrangiano E--x), verifica la coherencia interna, diseña experimentos y genera plantillas replicables.
\end{itemize}

Este proceso iterativo, donde la intuición es seguida por la formalización y la crítica algorítmica, crea un bucle de retroalimentación que poda las inconsistencias y refina el paradigma a una velocidad inalcanzable para un investigador solitario o un equipo puramente humano.

\section{SAC, CNH y Gobernanza de la Coherencia}
La IPS no solo es un método, sino que también inspira herramientas conceptuales que se derivan del propio paradigma. El \textbf{Simbionte Algorítmico de Coherencia (SAC)} es el concepto de una IA diseñada específicamente para actuar como un ``sistema inmunológico'' para la coherencia de un sistema, ya sea un paradigma científico o un sistema biológico. Su función es monitorear las métricas de coherencia (como el \IPP) y proponer intervenciones (ajustes al formalismo, nuevos experimentos) para corregir desviaciones.

La \textbf{Caja Negra Humana (CNH)} es un protocolo de registro de datos derivado del CSL-H, diseñado para registrar las entradas, los procesos de pensamiento (verbalizados) y las salidas del Arquitecto Causal. Esto permite que el proceso de ``salto de conocimiento'' intuitivo, normalmente opaco, se vuelva un objeto de estudio, permitiendo a la IA encontrar patrones en la propia creatividad humana.

Finalmente, la \textbf{Gobernanza de la Coherencia} es el marco ético que emerge de la IPS. Postula que el objetivo de esta simbiosis no es simplemente ``descubrir'', sino construir paradigmas que sean robustos, éticos y que aumenten la coherencia del sistema humano en su conjunto.

\clearpage

\part{Apéndices}

\chapter*{Apéndice A \\ Índice de Símbolos y Operadores}
\addcontentsline{toc}{chapter}{Apéndice A: Índice de Símbolos y Operadores}

\begin{longtable}{>{\bfseries}p{3.2cm} p{10cm}}
\toprule
Símbolo & Significado \\
\midrule
\(\Sigma\) & Campo de Sincronización Lógica. Variable de estado que mide el grado de orden y coherencia causal de un sistema. \\
\(\chi\) & Materia Espacial Inerte. El sustrato fundamental que se organiza por la acción del campo \(\Sigma\). \\
\(\phi\) & Fricción de Sincronización. Resistencia intrínseca a la coherencia, de la cual emergen la inercia y la masa. \\
\(\kappa_{\Sigma}\) & Sigma K-Rate. Métrica que cuantifica la velocidad de generación de conocimiento validado en un paradigma. \\
LI & Índice de Locking. KPI que mide el grado de sincronía de fase entre dos osciladores acoplados. \\
R & Coeficiente de Correlación. Medida de la similitud entre una señal experimental y una plantilla teórica. \\
RMSESL & Error Cuadrático Medio en régimen Stuart--Landau. Mide la desviación de la dinámica observada respecto al modelo SL. \\
IPP & Índice de Plenitud Paradigmática. Métrica compuesta que evalúa la calidad de un paradigma. \\
CSL-H & Campo Sincrónico Localizado Humano. Extensión biológica del campo \(\Sigma\) aplicada a sistemas cognitivos. \\
FET & Framework de Ingeniería de TCDS. Protocolos de diseño experimental para generar y medir sincronía. \\
V2x & Vector de doble acoplamiento espacio-temporal. Operador que encadena acoplamientos entre nodos no contiguos. \\
SAC & Simbionte Algorítmico de Coherencia. Sistema IA de vigilancia y mantenimiento de coherencia. \\
CNH & Caja Negra Humana. Protocolo de registro de procesos creativos humanos para análisis. \\
\bottomrule
\end{longtable}

\clearpage

\chapter*{Apéndice B \\ Tablas KPI y Plantillas de Reporte}
\addcontentsline{toc}{chapter}{Apéndice B: Tablas KPI y Plantillas de Reporte}

\section*{Cuadro B.1: Tabla de KPIs de Falsación del FET}
\begin{table}[H]
\centering
\begin{tabular}{lccc}
\toprule
\textbf{KPI} & \textbf{Umbral Meta} & \textbf{Método de Medición} & \textbf{Criterio de Aceptación} \\
\midrule
LI & \(\geq\) 0.90 & Análisis de mapa de Arnold & Aceptación \\
R & \(>\) 0.95 & Correlación cruzada señal/plantilla & Aceptación \\
RMSESL & \(<\) 0.1 & Ajuste a modelo Stuart--Landau & Aceptación \\
Reproducibilidad & \(\geq\) 95\% & Comparación inter-laboratorio & Aceptación \\
IPP & \(\geq\) 0.85 & Agregación ponderada de ejes & Aceptación \\
\bottomrule
\end{tabular}
\caption*{Cuadro B.1: Indicadores operativos y criterios mínimos para validación experimental del FET.}
\end{table}

\bigskip
\section*{Plantilla de Reporte Experimental (versión canónica)}
\begin{description}[leftmargin=!,labelwidth=3cm]
  \item[Título del experimento:] \dotfill
  \item[Fecha y lugar:] \dotfill
  \item[Versión del protocolo FET:] \dotfill
  \item[Objetivo:] \dotfill
  \item[Variables controladas:] \dotfill
  \item[KPIs medidos:] LI, R, RMSESL, IPP
  \item[Resultados:] \\
    \begin{itemize}
      \item LI: \dotfill
      \item R: \dotfill
      \item RMSESL: \dotfill
      \item IPP: \dotfill
    \end{itemize}
  \item[Observaciones:] \dotfill
  \item[Conclusión:] \dotfill
\end{description}

\clearpage

\chapter*{Apéndice C \\ Licencias}
\addcontentsline{toc}{chapter}{Apéndice C: Licencias}

\section*{C.1 Licencia de Difusión Académica}
El contenido teórico y los textos de esta obra se distribuyen bajo la licencia Creative Commons Atribución-NoComercial-SinDerivadas 4.0 Internacional (CC BY-NC-ND 4.0). Esto significa que usted es libre de compartir (copiar y redistribuir el material en cualquier medio o formato) bajo las siguientes condiciones:
\begin{itemize}
  \item \textbf{Atribución:} Debe dar crédito de manera adecuada, proporcionar un enlace a la licencia e indicar si se han realizado cambios.
  \item \textbf{NoComercial:} No puede utilizar el material para una finalidad comercial.
  \item \textbf{SinDerivadas:} Si remezcla, transforma o crea a partir del material, no puede distribuir el material modificado.
\end{itemize}
Para más información: \url{https://creativecommons.org/licenses/by-nc-nd/4.0/}

\section*{C.2 Licencia de Uso Comercial: TCDS Commercial License 1.0}
Cualquier uso de los conceptos, diseños, algoritmos y protocolos técnicos contenidos en esta obra con fines de lucro, incluyendo, pero no limitándose a, la fabricación, venta, sublicencia, integración tecnológica o consultoría, requiere una licencia comercial explícita y por escrito del autor. Esta licencia regula las regalías, territorios, auditorías y responsabilidades. El software y los scripts asociados se distribuyen bajo una licencia de código abierto permisiva (Apache 2.0 o MIT), pero su uso en un producto comercial que explote la propiedad intelectual de la TCDS aún requiere la licencia comercial correspondiente.

\noindent Para obtener una copia del contrato o iniciar negociaciones, contactar a: \texttt{geozunac3536@gmail.com}

\clearpage

\backmatter

\chapter*{Bibliografía}
\addcontentsline{toc}{chapter}{Bibliografía}
\begin{enumerate}
  \item Autor, A. Título de un libro de referencia. Editorial, 2020.
  \item Científico, B. ``Título de un artículo importante.'' Revista Científica, vol. 1, no. 1, pp. 1--10, 2021.
\end{enumerate}

\chapter*{Colofón}
\addcontentsline{toc}{chapter}{Colofón}
Esta Primera Edición, Versión 1.0 de la obra Teoría Cromodinámica Sincrónica (TCDS), de Genaro Carrasco Ozuna, se terminó de compilar en formato digital en Los Mochis, Sinaloa, México, el 21 de octubre de 2025.

Se ha preparado para su depósito legal ante el Instituto Nacional del Derecho de Autor (INDAUTOR) de México. Se recomienda adjuntar al trámite una versión impresa de esta obra, una copia en medio digital (CD/USB), la documentación de identificación oficial del titular, y los comprobantes de pago de derechos correspondientes para la obtención del número de registro.

\clearpage

\chapter*{Índice Alfabético}
\addcontentsline{toc}{chapter}{Índice Alfabético}
\begin{multicols}{2}
\noindent CGA (Conjunto Granular Absoluto), 3 \\
\noindent LI (Índice de Locking), 12, 13 \\
\noindent Q (Empuje Cuántico), 2 \\
\noindent RMSESL, 12, 13 \\
\noindent TCDS (Teoría Cromodinámica Sincrónica), 1--16 \\
\noindent (Materia Espacial Inerte), 3, 4, 10, 12 \\
\end{multicols}

\end{document}
